\documentclass[11pt,a4paper]{colorart}


\usepackage{amsmath}
\usepackage{amsfonts,amssymb}
\usepackage{tikz-cd}
\usepackage{hyperref}
\usepackage{quiver}

\def\maa{\mathcal}
\def\mab{\mathfrak}
\def\mac{\mathbb}
\def\rar{\rightarrow}
\def\tx{\times}
\def\oo{\circ}
\def\ra{\rightarrow}

\title{Category theory Notes}

\author{Raakesh M, 21MS116}
\date{updated on \today}

\begin{document}

\maketitle
\tableofcontents
\newpage

\section{Preamble}

Notes for category theory following Serge Lang's Algebra, section 11 of chapter 1.

\section{Basic definitions and examples}

\subsection{Category}

\begin{definition}[Category]
	%\newline
	A category $\mab{a}$ consists of,
	\begin{itemize}
		\item Collection of objects ob$(\mab{a})$,
		\item A set Mor$(A,B)$ for any two objects $A,B \in \text{ob}(\mab{a})$ called the morphisms of $A$ into $B$,
		\item And a "Law of composition" $L$ for any three objects $A,B,C \in \text{ob}(\mab{a})$, ie: a map,
		\[ L : \text{Mor}(B,C) \tx \text{Mor}(A,B) \rar \text{Mor}(A,C). \]
	\end{itemize}

	(If there arises no confusion $L(f,g)$ will be simply denoted as $f\oo g$.)

	Satisfying the following axioms:

	\begin{itemize}
		\item \textit{CAT 1} Two sets Mor$(A,B)$ and Mor$(A^\prime,B^\prime)$ are disjoint unless $A=A^\prime$ and $B=B^\prime$.
		\item \textit{CAT 2} For each object $A$, there is a morphism Id$_A \in \text{Mor}(A,A)$ which acts as left identity and right identity for the elements of Mor$(A,B)$ and Mor$(B,A)$ respectively for all $B \in \text{ob}(\mab{a})$.
		\item \textit{CAT 3} The law of composition is associative (when defined), ie: given $f \in \text{Mor}(A,B)$, $g \in \text{Mor}(B,C))$ and $h \in \text{Mor}(C,D)$ we have:
			\[ (f \oo g) \oo h = f \oo (g \oo h) \]
			for all $A,B,C,D \in \text{ob}(\mab{a})$.
	\end{itemize}

\end{definition} 

We shall denote collection of all morphisms in a category as $\text{Ar}(\mab{a})$, and "$f \in \text{Ar}(\mab{a})$" shall mean $f \in \text{Mor}(A,B)$ is a morphism for some $A,B \in \text{Ob}(\mab{a})$. We shall also refer to the collection of objects of a category as the category itself.

\begin{definition}[Isomorphism]
	$f \in \text{Mor}(A,B)$ is called an isomorphism if there exists $g \in \text{Mor}(B,A)$ such that $f \oo g = \text{Id}_B $ and $g \oo f = \text{Id}_A $, Set of all isomorphisms of $A$ into $B$ is represented by Iso$(A,B)$.
\end{definition}

\begin{definition}[Endomorphism]
Any morphism of an object into itself is called an endomorphism.
\end{definition}

\begin{definition}[Automorphism]
	Any endomorphism which is also an isomorphism is called an automorphism. Set of all automorphisms of an object $A$ is represented as Aut$(A)$.
\end{definition}

\begin{proposition}
	Aut$(A)$ forms a group with law of composition as the binary operator.
\end{proposition}

\begin{proof}
	Associativity of the binary operator is satisfied by definition, Id$_A$ acts as the identity element and for a morphism $f$, its inverse is the morphism $g$ such that $f \oo g = \text{Id}_A$. \\
	Therefore all group axioms are satisfied.
\end{proof}

\begin{example}[Category of sets]
	Let $\mab{S}$ be a category whose objects are sets and let the morphisms be maps between sets. \textit{CAT 1} is clearly true, \textit{CAT 2} is true as we let identity map be Id$_A$ for set $A$ and \textit{CAT 3} is also true as composition of maps clearly follow associativity.
\end{example}

\begin{example}[Category of groups]
	Let Grp be a category whose objects are groups and let the morphisms be group homomorphisms between groups.
\end{example}

\begin{example}[Category of monoids]
	Let Grp be a category whose objects are groups and let the morphisms be monoid homomorphisms between groups.
\end{example}

\begin{example}[$G$-sets]
	Given a group $G$ a $G$-set is a set $X$ with a map $\sigma : G \ra \text{S}(X)$ from group $G$ to the symmetric group of $X$. \\
	Category of $G$-sets is a category whose objects are $G$-sets and set of morphisms of $X$ into $X^\prime$ with corresponding maps $\sigma$ and $\sigma^\prime$ are maps $f : X \ra X^\prime$ such that $f(\sigma(g)(x)) = \sigma^\prime(g)(f(x))$ for all $x \in X$ and for all $g \in G$.
\end{example}

\begin{definition}[Operation]
	An operation of a group $G$ on an object $A$ is a group isomorphism of $G$ into the group Aut$(A)$.
\end{definition}

An operation of $G$ on $A$ is also called a representation of $G$ on $A$. \\

\subsection{Automorphisms operating on Isomorphisms}

We can also say  the group Aut$(B)$ operates on Iso$(A,B)$ via "law of composition", ie:

\[ \sigma: \text{Aut}(B) \ra \text{S}(\text{Iso}(A,B)) \]

where S$($Iso$(A,B))$ is the symmetric group of Iso$(A,B)$ and $\sigma(p)(q) = p \oo q$.\\

If $u_\oo$ and $u$ are in Iso$(A,B)$ then there exists $u^{-1}_\oo$ in Iso$(B,A)$ such that $u^{-1}_\oo \oo u_\oo = \text{Id}_B \in \text{Aut}(B)$ as $u_\oo$ is an isomorphism, then 

\[  u \oo u^{-1}_\oo \oo u_\oo = u \]

\[ \implies ( u \oo u^{-1}_\oo ) \oo u_\oo = u \]

\[ \implies \sigma ( u \oo u^{-1}_\oo ) ( u_\oo ) = u \]

Therefore we have, $\text{Orb}_{\text{Aut}(B)} u_\oo = \text{Iso}(A,B)$ and $\varphi: \text{Aut}(B) \ra \text{Iso}(A,B), u \ra u \oo u_\oo$ is a bijection where the inverse is given by $\varphi^{-1}: \text{Iso}(A,B) \ra \text{Aut}(B), u \ra u \oo u^{-1}_\oo$.\\

Similarly with Aut$(A)$ composing on the other side we get if there exists $u \in \text{Iso}(A,B)$ then Aut$(A)$ and Aut$(B)$ are isomorphic, namely 

\[ \phi : \text{Aut}(A) \ra \text{Aut}(B), \phi(w) = u \oo w \oo u^{-1} \]

is a group isomorphism and above can be visualized with the commutative diagram.\\

% https://q.uiver.app/#q=WzAsNCxbMCwwLCJBIl0sWzIsMCwiQiJdLFsyLDIsIkIiXSxbMCwyLCJBIl0sWzAsMywidyIsMl0sWzEsMiwidSBcXGNpcmMgdyBcXGNpcmMgdV57LTF9Il0sWzAsMSwidSJdLFszLDIsInUiLDJdXQ==
\[\begin{tikzcd}
	A && B \\
	\\
	A && B
	\arrow["w"', from=1-1, to=3-1]
	\arrow["{u w u^{-1}}", from=1-3, to=3-3]
	\arrow["u", from=1-1, to=1-3]
	\arrow["u"', from=3-1, to=3-3]
\end{tikzcd}\]

\subsection{Category of Representations}

Given a group $G$ let $\rho : G \ra \text{Aut}(A)$ and $\rho : G \ra \text{Aut}(A^\prime)$ be representation of group $G$ on 2 objects $A$ and $A^\prime$ of category $\mab{a}$.\\ 

Let $\mab{a}_G$ be category whose objects are representation of group $G$ on objects of $\mab{a}$ like the ones given above, let morphisms between representations be morphisms like $h$ in $\mab{a}$ which makes the following diagram commutative for all $x\in G$.

% https://q.uiver.app/#q=WzAsNCxbMCwwLCJBIl0sWzIsMCwiQV5cXHByaW1lIl0sWzIsMiwiQV5cXHByaW1lIl0sWzAsMiwiQSJdLFswLDMsIlxccmhvKHgpIiwyXSxbMywyLCJoIiwyXSxbMSwyLCJcXHJob15cXHByaW1lKHgpIl0sWzAsMSwiaCJdXQ==
\[\begin{tikzcd}
	A && {A^\prime} \\
	\\
	A && {A^\prime}
	\arrow["{\rho(x)}"', from=1-1, to=3-1]
	\arrow["h"', from=3-1, to=3-3]
	\arrow["{\rho^\prime(x)}", from=1-3, to=3-3]
	\arrow["h", from=1-1, to=1-3]
\end{tikzcd}\]

Indexing $h$ with $\rho$ and $\rho^\prime$ ensures \textit{CAT 1} is satisfied and it is easy to verify \textit{CAT 2} is also satisfied with the Identity morphism in $\mab{a}$ acting as the identity morphism from a representation to itself as the following diagram commutes for any automorphism $f$ of $A$ into itself.

\[\begin{tikzcd}
	A && {A} \\
	\\
	A && {A}
	\arrow["f"', from=1-1, to=3-1]
	\arrow["{\text{Id}_A}"', from=3-1, to=3-3]
	\arrow["f", from=1-3, to=3-3]
	\arrow["{\text{Id}_A}", from=1-1, to=1-3]
\end{tikzcd}\]

To verify \textit{CAT 3} let's start with 3 objects of $\mab{a}_G$ and morphisms between them, ie the following diagrams are commutative for all $x \in G$,

% https://q.uiver.app/#q=WzAsOCxbMCwwLCJBIl0sWzIsMCwiQV5cXHByaW1lIl0sWzIsMiwiQV5cXHByaW1lIl0sWzAsMiwiQSJdLFs0LDAsIkFeXFxwcmltZSJdLFs0LDIsIkFeXFxwcmltZSJdLFs2LDAsIkFee1xccHJpbWUgXFxwcmltZX0iXSxbNiwyLCJBXntcXHByaW1lIFxccHJpbWV9Il0sWzAsMywiXFxyaG8oeCkiLDJdLFszLDIsImhfMSIsMl0sWzEsMiwiXFxyaG9eXFxwcmltZSh4KSJdLFswLDEsImhfMSJdLFs0LDUsIlxccmhvXlxccHJpbWUoeCkiLDJdLFs2LDcsIlxccmhvXntcXHByaW1lIFxccHJpbWV9KHgpIl0sWzQsNiwiaF8yIl0sWzUsNywiaF8yIiwyXV0=
\[\begin{tikzcd}
	A && {A^\prime} && {A^\prime} && {A^{\prime \prime}} \\
	\\
	A && {A^\prime} && {A^\prime} && {A^{\prime \prime}}
	\arrow["{\rho(x)}"', from=1-1, to=3-1]
	\arrow["{h_1}"', from=3-1, to=3-3]
	\arrow["{\rho^\prime(x)}", from=1-3, to=3-3]
	\arrow["{h_1}", from=1-1, to=1-3]
	\arrow["{\rho^\prime(x)}"', from=1-5, to=3-5]
	\arrow["{\rho^{\prime \prime}(x)}", from=1-7, to=3-7]
	\arrow["{h_2}", from=1-5, to=1-7]
	\arrow["{h_2}"', from=3-5, to=3-7]
\end{tikzcd}\]

Now we have,

% https://q.uiver.app/#q=WzAsNixbMCwwLCJBIl0sWzIsMCwiQV5cXHByaW1lIl0sWzIsMiwiQV5cXHByaW1lIl0sWzAsMiwiQSJdLFs0LDAsIkFee1xccHJpbWVcXHByaW1lfSJdLFs0LDIsIkFee1xccHJpbWVcXHByaW1lfSJdLFswLDMsIlxccmhvKHgpIiwyXSxbMywyLCJoXzEiLDJdLFsxLDIsIlxccmhvXlxccHJpbWUoeCkiXSxbMCwxLCJoXzEiXSxbNCw1LCJcXHJob157XFxwcmltZSBcXHByaW1lfSh4KSJdLFsyLDUsImhfMiIsMl0sWzEsNCwiaF8yIl1d
\[\begin{tikzcd}
	A && {A^\prime} && {A^{\prime\prime}} \\
	\\
	A && {A^\prime} && {A^{\prime\prime}}
	\arrow["{\rho(x)}"', from=1-1, to=3-1]
	\arrow["{h_1}"', from=3-1, to=3-3]
	\arrow["{\rho^\prime(x)}", from=1-3, to=3-3]
	\arrow["{h_1}", from=1-1, to=1-3]
	\arrow["{\rho^{\prime \prime}(x)}", from=1-5, to=3-5]
	\arrow["{h_2}"', from=3-3, to=3-5]
	\arrow["{h_2}", from=1-3, to=1-5]
\end{tikzcd}\]

And hence the following diagram is also commutative,

% https://q.uiver.app/#q=WzAsNCxbMCwwLCJBIl0sWzAsMiwiQSJdLFs0LDAsIkFee1xccHJpbWVcXHByaW1lfSJdLFs0LDIsIkFee1xccHJpbWVcXHByaW1lfSJdLFswLDEsIlxccmhvKHgpIiwyXSxbMiwzLCJcXHJob157XFxwcmltZSBcXHByaW1lfSh4KSJdLFswLDIsImhfMSBcXGNpcmMgaF8yIl0sWzEsMywiaF8xIFxcY2lyYyBoXzIiLDJdXQ==
\[\begin{tikzcd}
	A &&&& {A^{\prime\prime}} \\
	\\
	A &&&& {A^{\prime\prime}}
	\arrow["{\rho(x)}"', from=1-1, to=3-1]
	\arrow["{\rho^{\prime \prime}(x)}", from=1-5, to=3-5]
	\arrow["{h_1 \circ h_2}", from=1-1, to=1-5]
	\arrow["{h_1 \circ h_2}"', from=3-1, to=3-5]
\end{tikzcd}\]

Therefore law of composition of $\mab{a}_G$ is simple composing morphisms in $\mab{a}$.

Let us now cosider the category Ab, the category of abelian groups, let $A$ be an abelian group and $G$ any group, consider a group homomorphism $\rho : G \ra \text{Aut}(A)$, ie: a representation of $G$ on object $A$ in thecategory of abelian groups,\\

let us represent $\rho(x)(a)$ with $x . a$, \\

let $x,y \in G$ and $a,b \in A$, then we have,

\[ x.(y.a) = (xy).a \]

\[ x.(a+b) = x.a + x.b \]

\[ e.a = a \]

\[ x.0 = 0 \]

where $+$ is the binary operation of $A$ and identity elements of $G$ and $A$ are $e$ and $0$ repectively.\\

When $G$ acts on itself by conjugation, $G$ operates on itself as a set as well as in the category of groups, for each $x\in G$ corresponds an operation $\Phi$ such that,

\[ \Phi : G \ra \mab{O}, x \ra \rho(x \_) \]

where $\mab{O}$ is the set of all representation of $G$ on object $A$ in the category of abelian groups,

\[ \rho(x\_) : G \ra \text{Aut}(A), g \ra \rho(xy) \in \text{Aut}(A). \]

\subsection{Category of Morphisms}

Let $\mab{a}$ be a category and let $\mab{c}$ be a category whose objects are the morphisms of $\mab{a}$, if $f \in \text{Mor}(A,B)$ $f^\prime \in \text{Mor}(A^\prime,B^\prime)$ be two objects in $\mab{c}$ then let morphisms in $\mab{c}$ of $f$ into $f^\prime$ be ordered pair $(\psi, \varphi)$\\

where $\psi$ and $\varphi$ are morphisms in $\mab{a}$, $\psi \in \text{Mor}(A,A^\prime)$, $\varphi \in \text{Mor}(B,B^\prime)$, and makes the following diagram commutative,

% https://q.uiver.app/#q=WzAsNCxbMCwwLCJBIl0sWzIsMCwiQV5cXHByaW1lIl0sWzAsMiwiQiJdLFsyLDIsIkJeXFxwcmltZSJdLFswLDIsImYiLDJdLFsxLDMsImZeXFxwcmltZSJdLFswLDEsIlxccHNpIl0sWzIsMywiXFx2YXJwaGkiLDJdXQ==
\[\begin{tikzcd}
	A && {A^\prime} \\
	\\
	B && {B^\prime}
	\arrow["f"', from=1-1, to=3-1]
	\arrow["{f^\prime}", from=1-3, to=3-3]
	\arrow["\psi", from=1-1, to=1-3]
	\arrow["\varphi"', from=3-1, to=3-3]
\end{tikzcd}\]

To satisfy \textit{CAT 1} the ordered pair must be indexed with $f$ and $f^\prime$, to satisfy \textit{CAT 2} it is easy to verify $($Id$_A$,Id$_B)$ satisfies as the identity morphism, ie the following diagram commutes,

\[\begin{tikzcd}
	A && {A} \\
	\\
	B && {B}
	\arrow["f"', from=1-1, to=3-1]
	\arrow["{f}", from=1-3, to=3-3]
	\arrow["{\text{Id}_A}", from=1-1, to=1-3]
	\arrow["{\text{Id}_B}"', from=3-1, to=3-3]
\end{tikzcd}\]

To see if \textit{CAT 3} is satisfied let us start with 3 objects of $\mab{c}$ and 2 morphisms between them, ie, the following diagrams are commutative,

% https://q.uiver.app/#q=WzAsOCxbMCwwLCJBIl0sWzIsMCwiQV5cXHByaW1lIl0sWzAsMiwiQiJdLFsyLDIsIkJeXFxwcmltZSJdLFs0LDAsIkFeXFxwcmltZSJdLFs0LDIsIkJeXFxwcmltZSJdLFs2LDAsIkFee1xccHJpbWVcXHByaW1lfSJdLFs2LDIsIkJee1xccHJpbWVcXHByaW1lfSJdLFswLDIsImYiLDJdLFsxLDMsImZeXFxwcmltZSJdLFswLDEsIlxccHNpIl0sWzIsMywiXFx2YXJwaGkiLDJdLFs0LDUsImZeXFxwcmltZSIsMl0sWzYsNywiZl57XFxwcmltZVxccHJpbWV9Il0sWzQsNiwiXFxwc2leXFxwcmltZSJdLFs1LDcsIlxcdmFycGhpXlxccHJpbWUiLDJdXQ==
\[\begin{tikzcd}
	A && {A^\prime} && {A^\prime} && {A^{\prime\prime}} \\
	\\
	B && {B^\prime} && {B^\prime} && {B^{\prime\prime}}
	\arrow["f"', from=1-1, to=3-1]
	\arrow["{f^\prime}", from=1-3, to=3-3]
	\arrow["\psi", from=1-1, to=1-3]
	\arrow["\varphi"', from=3-1, to=3-3]
	\arrow["{f^\prime}"', from=1-5, to=3-5]
	\arrow["{f^{\prime\prime}}", from=1-7, to=3-7]
	\arrow["{\psi^\prime}", from=1-5, to=1-7]
	\arrow["{\varphi^\prime}"', from=3-5, to=3-7]
\end{tikzcd}\]

Now we have,

% https://q.uiver.app/#q=WzAsNixbMCwwLCJBIl0sWzAsMiwiQiJdLFs0LDIsIkJee1xccHJpbWVcXHByaW1lfSJdLFs0LDAsIkFee1xccHJpbWVcXHByaW1lfSJdLFsyLDIsIkJeXFxwcmltZSJdLFsyLDAsIkFeXFxwcmltZSJdLFswLDEsImYiLDJdLFszLDIsImZee1xccHJpbWVcXHByaW1lfSJdLFsxLDQsIlxcdmFycGhpIiwyXSxbNSw0LCJmXlxccHJpbWUiXSxbMCw1LCJcXHBzaSJdLFs1LDMsIlxccHNpXlxccHJpbWUiXSxbNCwyLCJcXHZhcnBoaV5cXHByaW1lIiwyXV0=
\[\begin{tikzcd}
	A && {A^\prime} && {A^{\prime\prime}} \\
	\\
	B && {B^\prime} && {B^{\prime\prime}}
	\arrow["f"', from=1-1, to=3-1]
	\arrow["{f^{\prime\prime}}", from=1-5, to=3-5]
	\arrow["\varphi"', from=3-1, to=3-3]
	\arrow["{f^\prime}", from=1-3, to=3-3]
	\arrow["\psi", from=1-1, to=1-3]
	\arrow["{\psi^\prime}", from=1-3, to=1-5]
	\arrow["{\varphi^\prime}"', from=3-3, to=3-5]
\end{tikzcd}\]

And hence,

% https://q.uiver.app/#q=WzAsNCxbMCwwLCJBIl0sWzAsMiwiQiJdLFs0LDIsIkJee1xccHJpbWVcXHByaW1lfSJdLFs0LDAsIkFee1xccHJpbWVcXHByaW1lfSJdLFswLDEsImYiLDJdLFszLDIsImZee1xccHJpbWVcXHByaW1lfSJdLFswLDMsIlxccHNpIFxcY2lyYyBcXHBzaV5cXHByaW1lIl0sWzEsMiwiXFx2YXJwaGlcXGNpcmNcXHZhcnBoaV5cXHByaW1lIiwyXV0=
\[\begin{tikzcd}
	A &&&& {A^{\prime\prime}} \\
	\\
	B &&&& {B^{\prime\prime}}
	\arrow["f"', from=1-1, to=3-1]
	\arrow["{f^{\prime\prime}}", from=1-5, to=3-5]
	\arrow["{\psi \circ \psi^\prime}", from=1-1, to=1-5]
	\arrow["{\varphi\circ\varphi^\prime}"', from=3-1, to=3-5]
\end{tikzcd}\]

Therefore $(\psi,\varphi) \oo (\psi^\prime,\varphi^\prime) = (\psi \oo \psi ^ \prime,\varphi \oo \varphi ^ \prime)$ in the category of morphisms.\\

Similarly we can construct other categories, where we fix objects to be morphisms of fixed arrival or departure. For given category $\mab{a}$ we denote category of morphisms with fixed arrival $A$ as $\mab{a}_A$ and for fixed departure $A$ we denote it as $\mab{a}^A$.\\


For $f\in \text{Mor}(X,A)$ and $g\in \text{Mor}(Y,A)$, morphisms of $f$ into $g$ are morphisms $h \in \text{Ar}(\mab{a})$ such that the following diagram commutes,

% https://q.uiver.app/#q=WzAsMyxbMCwwLCJYIl0sWzIsMCwiWSJdLFsxLDIsIkEiXSxbMCwyLCJmIiwyXSxbMSwyLCJnIl0sWzAsMSwiaCJdXQ==
\[\begin{tikzcd}
	X && Y \\
	\\
	& A
	\arrow["f"', from=1-1, to=3-2]
	\arrow["g", from=1-3, to=3-2]
	\arrow["h", from=1-1, to=1-3]
\end{tikzcd}\]

Here $f$ and $g$ are objects in $\mab{a}_A$.\\

For $p\in \text{Mor}(X,A)$ and $q\in \text{Mor}(Y,A)$, morphisms of $p$ into $q$ are morphisms $h \in \text{Ar}(\mab{a})$ such that the following diagram commutes,

% https://q.uiver.app/#q=WzAsMyxbMCwwLCJYIl0sWzIsMCwiWSJdLFsxLDIsIkEiXSxbMCwxLCJoIl0sWzIsMCwicCJdLFsyLDEsInEiLDJdXQ==
\[\begin{tikzcd}
	X && Y \\
	\\
	& A
	\arrow["h", from=1-1, to=1-3]
	\arrow["p", from=3-2, to=1-1]
	\arrow["q"', from=3-2, to=1-3]
\end{tikzcd}\]

Here $p$ and $q$ are objects in $\mab{a}^A$.\\

\section{Universal Objects}

\begin{definition}[Universal attractor]
	Let $\mab{c}$ be a category, $p$ an object in $\mab{c}$ is said to be universally attracting if the set Mor$(X,p)$ is a singleton set for all $X \in \text{Ob}(\mab{c})$. 
\end{definition}

\begin{definition}[Universal repeller]
	Let $\mab{c}$ be a category, $p$ an object in $\mab{c}$ is said to be universally repelling if the set Mor$(p,X)$ is a singleton set for all $X \in \text{Ob}(\mab{c})$. 
\end{definition}

If there is no ambiguity we refer to both of the above as universal objects.

\begin{proposition}[]
	If $p$ is universally attracting and $q$ is universally repelling then there is an isomorphism between them.
\end{proposition}

\begin{proof}
	Let $f$ be the unique morphism of $p$ into $q$, and let $g$ be the unique morphism of $q$ into $p$. $f \oo g$ is a morphism of $q$ into itself and since $q$ is univerally repelling, it is unique and hence the identity morphism and hence $f$ is an isomorphism and similarly for $g$.
\end{proof}

\begin{example}[Trivial group]
	$\{e\}$ is both universally repelling and universally attracting in the category of groups as the only group homomorphism from and to $\{e\}$ is the constant function to the identity element.
\end{example}

%\begin{example}[]
%	Let $S$ be a set, let $\mab{c}$ be the category of maps $f : S \ra A$ where $A$ is an abelian group. Let $f : S \ra A$ and $f^\prime : S \ra A^\prime$ be 2 objects in $\mab{c}$, then elements of Mor$(f,f^\prime)$ are group homomorphisms $h : A \ra A^\prime$ such that the following diagram commutes,

%\end{example}

\section{Products and Coproducts}

\subsection{Product}

\begin{definition}[Product of 2 objects]
	Let $\mab{a}$ be a category and let $A$ and $B$ be 2 objects, ordered pair $(P,\{f,g\})$ is said to be product of $A$ and $B$ if $P$ is an object of $\mab{a}$, $f \in \text{Mor}(P,A)$ and $g \in \text{Mor}(P,B)$ and follow the following property,\\

	Given $\psi \in \text{Mor}(C,A)$ and $\phi \in \text{Mor}(C,B)$ (where $C$ is an object in $\mab{a}$), the there exist an unique $h \in \text{Mor}(C,P)$ such that the following diagram commutes,

% https://q.uiver.app/#q=WzAsNCxbMiwyLCJQIl0sWzIsMCwiQyJdLFs0LDIsIkIiXSxbMCwyLCJBIl0sWzAsMiwiZyIsMl0sWzEsMiwiXFxwaGkiXSxbMSwwLCJcXGV4aXN0cyAhaCIsMSx7InN0eWxlIjp7ImJvZHkiOnsibmFtZSI6ImRhc2hlZCJ9fX1dLFsxLDMsIlxccHNpIiwyXSxbMCwzLCJmIl1d
\[\begin{tikzcd}
	&& C \\
	\\
	A && P && B
	\arrow["g"', from=3-3, to=3-5]
	\arrow["\phi", from=1-3, to=3-5]
	\arrow["{\exists !h}"', dashed, from=1-3, to=3-3]
	\arrow["\psi"', from=1-3, to=3-1]
	\arrow["f", from=3-3, to=3-1]
\end{tikzcd}\]


ie: $\psi = f \oo g$ and $\phi = g \oo h$. More generally we have,

\end{definition}

\begin{definition}[Product of a family of objects]
	Let $\mab{a}$ be a category and let $\{A_i\}_{i \in I}$ be a family of objects, ordered pair $(P,\{f_i\}_{i \in I})$ is said to be product of $\{A_i\}_{i \in I}$ if $P$ is an object of $\mab{a}$ and $f_i \in \text{Mor}(P,A_i)$ for all $i\in I$ and follow the following property,\\

	Given an object $C$ in $\mab{a}$ and family of morphisms $\{ g_i \in \text{Mor}(C,A_i) \}_{i\in I}$ there exist unique $h \in \text{Mor}(C,P)$ such that the following diagram commutes for all $i \in I$,

% https://q.uiver.app/#q=WzAsMyxbMCwwLCJDIl0sWzAsMSwiUCJdLFsyLDEsIkFfaSJdLFswLDEsIntcXGV4aXN0cyAhIGh9IiwyLHsic3R5bGUiOnsiYm9keSI6eyJuYW1lIjoiZGFzaGVkIn19fV0sWzAsMiwie2dfaX0iXSxbMSwyLCJ7Zl9pfSIsMl1d
\[\begin{tikzcd}
	C \\
	P && {A_i}
	\arrow["{{\exists ! h}}"', dashed, from=1-1, to=2-1]
	\arrow["{{g_i}}", from=1-1, to=2-3]
	\arrow["{{f_i}}"', from=2-1, to=2-3]
\end{tikzcd}\]

ie: $f_i \oo h = g_i$ for all $i \in I$.

\end{definition}

\begin{example}[Product of sets]
	Let $\mab{s}$ be the category of sets and let $\{a_i\}_{i \in I}$ be a family of sets, let $A$ be the cartesian product of the family of sets  $\{a_i\}_{i \in I}$ and let  $p_i : A \ra a_i$ be the projection on the i$^\text{th}$ factor. It is easy to verify that $(A, \{p_i\}_{i \in I})$ is a product of $\{a_i\}_{i \in I}$.
\end{example}

\begin{example}[Direct product of groups]
	Let $\{G_i\}_{i \in I}$ be a family of groups, recall the definition of direct product of groups, let $(G, \{\pi_i\}_{i \in I})$ be the direct product of the family of groups. By defintion of direct product, it is easy to see direct product of groups is the product in the category of groups.
\end{example}

\begin{example}[Product as universal objects]
	Let $\{A_i\}_{i \in I}$ be a family of objects in a category $\mab{a}$, let $\mab{c}$ be a category whose objects are $(X, \{f_i\}_{i \in I})$ where $X \in \text{Ob}(\mab{a}$ and $f_i \in \text{Mor}(X,A_i) \forall i \in I$ and morphisms of $(X, \{f_i\}_{i \in I})$ into $(Y, \{g_i\}_{i \in I})$ are morphisms $h \in \text{Mor}(X,Y)$ such that $f_i = g_i \oo h$ for all $i \in I$. \\

	We can see that product of family of objects $\{A_i\}_{i \in I}$ is an universal object in $\mab{c}$.
\end{example}

\subsection{Coproduct}

\begin{definition}[Coproduct of a family of objects]
	Let $\mab{a}$ be a category and let $\{A_i\}_{i \in I}$ be a family of objects, ordered pair $(S,\{f_i\}_{i \in I})$ is said to be product of $\{A_i\}_{i \in I}$ if $S$ is an object of $\mab{a}$ and $f_i \in \text{Mor}(A_i,S)$ for all $i\in I$ and follow the following property,\\

	Given an object $C$ in $\mab{a}$ and family of morphisms $\{ g_i \in \text{Mor}(A_i,C) \}_{i\in I}$ there exist unique $h \in \text{Mor}(S,C)$ such that the following diagram commutes for all $i \in I$,

% https://q.uiver.app/#q=WzAsNCxbMCwwLCJDIl0sWzAsMSwiUyJdLFsyLDEsIkFfaSJdLFsxLDJdLFsxLDAsInt7XFxleGlzdHMgISBofX0iLDAseyJzdHlsZSI6eyJib2R5Ijp7Im5hbWUiOiJkYXNoZWQifX19XSxbMiwwLCJ7e2dfaX19IiwyXSxbMiwxLCJ7e2ZfaX19Il1d
\[\begin{tikzcd}
	C \\
	S && {A_i} \\
	& {}
	\arrow["{{{\exists ! h}}}", dashed, from=2-1, to=1-1]
	\arrow["{{{g_i}}}"', from=2-3, to=1-1]
	\arrow["{{{f_i}}}", from=2-3, to=2-1]
\end{tikzcd}\]

ie: $h \oo f_i = g_i$ for all $i \in I$.

\end{definition}

\begin{example}[Coproduct of sets]
	Let $\mab{s}$ be the category of sets, let $s_1$ and $s_2$ be 2 sets and let $t$ be a set having the same cardinality as $s_2$ and disjoint from $s_1$, let $f_1 : s_1 \ra s_1$ be identity map and $f_2: s_2 \ra t$ be a bijection. Let $U$ be the union of $s_1$ and $t$, then $(U,\{g_1,g_2\})$ is coproduct of $s_1$ and $s_2$ where $g_i: s_i \ra U$ and $g_i(x) = f_i(x) \forall x \in s_i$.  
\end{example}

\begin{example}[]
	Let $\mab{s}_\oo$ be the category of pointed sets, pointed sets are ordered pairs $(s,x)$ where $x \in s$. Morphisms of $(s,x)$ into $(s^\prime,x^\prime)$ are maps $f: s \ra s^\prime$ such that $f(x) = x^\prime$. Coproduct of $(s,x)$ and $(s^\prime,x^\prime)$ can be constructed as follows,\\ 
	let $t$ be a set with the same cardinality as $s^\prime$ and $t \cap s = \{x\}$, let $U = t \cup s$ and let $f_1 : s \ra U, f_1(a) = a$ and $f_2: s^\prime \ra U$ where $f_2$ induces a bijection from $s^\prime - \{x\}$ to $t - \{x\}$ and now we have $(U, \{f_1, f_2\})$ as the co product.
\end{example}

\newpage

\begin{example}[Coproduct as universal objects]
	Let $\{A_i\}_{i \in I}$ be a family of objects in a category $\mab{a}$, let $\mab{c}$ be a category whose objects are $(X, \{f_i\}_{i \in I})$ where $X \in \text{Ob}(\mab{a}$ and $f_i \in \text{Mor}(A_i,X) \forall i \in I$ and morphisms of $(X, \{f_i\}_{i \in I})$ into $(Y, \{g_i\}_{i \in I})$ are morphisms $h \in \text{Mor}(X,Y)$ such that $h \oo f_i = g_i $ for all $i \in I$. \\

	We can see that coproduct of family of objects $\{A_i\}_{i \in I}$ is an universal object in $\mab{c}$.
\end{example}

\section{Fibered products and coproducts}

\begin{definition}[Category over $Z$]
	Let $\mab{c}$ be a category, and let $Z$ be an object category over $Z$ denoted by $\mab{c}_Z$ is the category whose objects are morphisms in $\mab{c}$ with fixed arrival $Z$ and whose morphisms are given as follows, morphisms of $f \in \text{Mor}(X,Z)$ into $g \in \text{Mor}(Y,Z)$ are morphisms $h \in \text{Mor}(X,Y)$ which makes the following diagram commutative,

% https://q.uiver.app/#q=WzAsMyxbMCwwLCJYIl0sWzIsMCwiWSJdLFsxLDIsIloiXSxbMCwxLCJoIl0sWzAsMiwiZiIsMl0sWzEsMiwiZyJdXQ==
\[\begin{tikzcd}
	X && Y \\
	\\
	& Z
	\arrow["h", from=1-1, to=1-3]
	\arrow["f"', from=1-1, to=3-2]
	\arrow["g", from=1-3, to=3-2]
\end{tikzcd}\]

ie: $h \oo g = f$.
\end{definition}

\begin{definition}[Fibered Product]
	A product in the category $\mab{c}_Z$ is called the fibered product of morphisms in $\mab{c}$ and is denoted with $X \times_Z Y$.
\end{definition}

Clarfying the above definition with commutative diagrams, \\

Fibered product of $f \in \text{Mor}(X,Z)$ and $g \in \text{Mor}(Y,Z)$ is $s \in \text{Mor}(X\times_ZY,Z)$ along with morphisms $p_1$ and $p_2$ such that following diagram commutes,

% https://q.uiver.app/#q=WzAsNCxbMCwwLCJYXFx0aW1lc19aWSJdLFswLDIsIlgiXSxbMiwwLCJZIl0sWzIsMiwiWiJdLFsxLDMsImYiLDJdLFsyLDMsImciXSxbMCwxLCJwXzEiLDJdLFswLDIsInBfMiJdLFswLDMsInMiLDEseyJzdHlsZSI6eyJib2R5Ijp7Im5hbWUiOiJkYXNoZWQifX19XV0=
\[\begin{tikzcd}
	{X\times_ZY} && Y \\
	\\
	X && Z
	\arrow["f"', from=3-1, to=3-3]
	\arrow["g", from=1-3, to=3-3]
	\arrow["{p_1}"', from=1-1, to=3-1]
	\arrow["{p_2}", from=1-1, to=1-3]
	\arrow["s"{description}, dashed, from=1-1, to=3-3]
\end{tikzcd}\]

(Here $p_1$ is called the pull back of $g$ by $f$ and $p_2$ is called the pull back of $f$ by $g$)

satisfying the following property, given $T$, an object in $\mab{c}$ and morphisms of $C$ into $X$ and $Y$ with the following diagram commutative,

% https://q.uiver.app/#q=WzAsNCxbMCwwLCJUIl0sWzAsMiwiWCJdLFsyLDAsIlkiXSxbMiwyLCJaIl0sWzEsMywiZiIsMl0sWzIsMywiZyJdLFswLDEsIngiLDJdLFswLDIsInkiXSxbMCwzLCJ0IiwxLHsic3R5bGUiOnsiYm9keSI6eyJuYW1lIjoiZGFzaGVkIn19fV1d
\[\begin{tikzcd}
	T && Y \\
	\\
	X && Z
	\arrow["f"', from=3-1, to=3-3]
	\arrow["g", from=1-3, to=3-3]
	\arrow["x"', from=1-1, to=3-1]
	\arrow["y", from=1-1, to=1-3]
	\arrow["t"{description}, dashed, from=1-1, to=3-3]
\end{tikzcd}\]

\newpage

Then there exists an unique morphisms $h$ of $T$ into $X\times_ZY$ and the following diagram commutes,

% https://q.uiver.app/#q=WzAsNSxbMSwxLCJYXFx0aW1lc19aWSJdLFsxLDMsIlgiXSxbMywxLCJZIl0sWzMsMywiWiJdLFswLDAsIlQiXSxbMSwzLCJmIiwyXSxbMiwzLCJnIl0sWzAsMSwicF8xIiwyXSxbMCwyLCJwXzIiXSxbNCwwLCJoIiwxXSxbNCwxLCJ4IiwyLHsiY3VydmUiOjJ9XSxbNCwyLCJ5IiwwLHsiY3VydmUiOi0yfV1d
\[\begin{tikzcd}
	T \\
	& {X\times_ZY} && Y \\
	\\
	& X && Z
	\arrow["f"', from=4-2, to=4-4]
	\arrow["g", from=2-4, to=4-4]
	\arrow["{p_1}"', from=2-2, to=4-2]
	\arrow["{p_2}", from=2-2, to=2-4]
	\arrow["h"{description}, from=1-1, to=2-2]
	\arrow["x"', curve={height=12pt}, from=1-1, to=4-2]
	\arrow["y", curve={height=-12pt}, from=1-1, to=2-4]
\end{tikzcd}\]


\begin{definition}[Category under $Z$]
	Let $\mab{c}$ be a category, and let $Z$ be an object, category undere $Z$ denoted by $\mab{c}^Z$ is the category whose objects are morphisms in $\mab{c}$ with fixed departure $Z$ and whose morphisms are given as follows, morphisms of $f \in \text{Mor}(Z,X)$ into $g \in \text{Mor}(Z,Y)$ are morphisms $h \in \text{Mor}(X,Y)$ which makes the following diagram commutative,

% https://q.uiver.app/#q=WzAsNCxbMSwwLCJYIl0sWzMsMCwiWSJdLFsyLDIsIloiXSxbMCwxXSxbMCwxLCJoIl0sWzIsMCwiZiJdLFsyLDEsImciLDJdXQ==
\[\begin{tikzcd}
	& X && Y \\
	{} \\
	&& Z
	\arrow["h", from=1-2, to=1-4]
	\arrow["f", from=3-3, to=1-2]
	\arrow["g"', from=3-3, to=1-4]
\end{tikzcd}\]

ie: $h \oo g = f$.
\end{definition}

\begin{definition}[Fibered Coproduct]
	A product in the category $\mab{c}Z$ is called the fibered product of morphisms in $\mab{c}$ and is denoted with $X \amalg_ZY$.
\end{definition}

Similar to how fibered product was expressed with commutativ diagram, fibered coproduct can visualized with the following commutative diagram,

% https://q.uiver.app/#q=WzAsNSxbMSwxLCJYXFxhbWFsZ19aWSJdLFsxLDMsIlgiXSxbMywxLCJZIl0sWzMsMywiWiJdLFswLDAsIlQiXSxbMywxLCJmIl0sWzMsMiwiZyIsMl0sWzEsMCwicV8xIl0sWzIsMCwicV8yIiwyXSxbNCwwLCJoIiwxXSxbMSw0LCJ4IiwwLHsiY3VydmUiOi0yfV0sWzIsNCwieSIsMix7ImN1cnZlIjoyfV1d
\[\begin{tikzcd}
	T \\
	& {X\amalg_ZY} && Y \\
	\\
	& X && Z
	\arrow["f", from=4-4, to=4-2]
	\arrow["g"', from=4-4, to=2-4]
	\arrow["{q_1}", from=4-2, to=2-2]
	\arrow["{q_2}"', from=2-4, to=2-2]
	\arrow["h"{description}, from=1-1, to=2-2]
	\arrow["x", curve={height=-12pt}, from=4-2, to=1-1]
	\arrow["y"', curve={height=12pt}, from=2-4, to=1-1]
\end{tikzcd}\]

Here $q_1$ is called the push out of $g$ by $f$ and $q_2$ is called the push back of $f$ by $g$.

\section{Functors}

\subsection{Definitions}

\begin{definition}[Covarient Functor]
	Let $\mab{a}, \mab{b}$ be categories. A covarient functor $F$ of $\mab{a}$ into $\mab{b}$ is a rule which associates to each object $A \in \mab{a}$ an object $F(A) \in \mab{b}$ and to each morphisms $f \in \text{Mor}(A,B)$ a morphism $F(f) \in \text{Mor}(F(A),F(B))$ while satisfying the following axioms,
	\begin{itemize}
		\item \textit{FUN 1} For all $A \in \mab{a}$ we have $F(\text{Id}_A) = \text{Id}_{F(A)}$.
		\item \textit{FUN 2} For $f \in \text{Mor}(A,B)$ and $g \in \text{Mor}(B,C)$ we have $F(g \oo f) = F(g) \oo F(f)$.
	\end{itemize}
\end{definition}

\newpage

\begin{definition}[Cotravarient Functor]
	Let $\mab{a}, \mab{b}$ be categories. A covarient functor $F$ of $\mab{a}$ into $\mab{b}$ is a rule which associates to each object $A \in \mab{a}$ an object $F(A) \in \mab{b}$ and to each morphisms $f \in \text{Mor}(A,B)$ a morphism $F(f) \in \text{Mor}(F(B),F(A))$ while satisfying the following axioms,
	\begin{itemize}
		\item \textit{FUN 1} For all $A \in \mab{a}$ we have $F(\text{Id}_A) = \text{Id}_{F(A)}$.
		\item \textit{FUN 2} For $f \in \text{Mor}(A,B)$ and $g \in \text{Mor}(B,C)$ we have $F(g \oo f) = F(f) \oo F(g)$.
	\end{itemize}
\end{definition}

\begin{example}[]
	Let $\mab{c}$ be a category and for $f \in \text{Mor}(A,B)$, we say $g \in \text{Mor}(B,A)$ is the section of $f$ if $f \oo g = \text{Id}_A$. Let $H$ be a functor from $\mab{c}$ to the category of groups, let $f \in \text{Mor}(A,B)$ then if $H(B) = \{e\}$ while $H(A) \neq \{e\}$, then there does not exist a section of $f$. Since, $H(g \oo f) = \text{Id}_{H(A)}$ and $H(f)$ is the trivial homorphism, which is the identity morphism in the category of groups.
\end{example}

\begin{example}[Representation Functor]
	Let $\mab{a}$ be a category and fix object $A$ in $\mab{a}$, then we obtain the covarient functor $M_A$ of $\mab{a}$ into $\mab{s}$ by letting $M_A(X) = \text{Mor}(A,X)$ for any object $X$ in $\mab{a}$ and if $\psi \in \text{Mor}(X,X^\prime)$ then $M_A(\varphi)(g) = \varphi \oo g$.\\
	Here $\mab{s}$ is a category whose objects are Mor$(A,X)$ and morphisms are $M_A(\phi)$ where $\phi$'s are morphisms in $\mab{a}$.\\
	Similarly for fixed object $B$ in $\mab{a}$, we obtain the cotravarient functor $M^B$ of $\mab{a}$ into $\mab{s^\prime}$ by letting $M^B(X) = \text{Mor}(X,B)$ for any object $X$ in $\mab{a}$ and if $\psi \in \text{Mor}(X,X^\prime)$ then $M_A(\psi)(g) = g \oo \psi$.\\
	For both of the functors it is easy to check that \textit{FUN 1} and \textit{FUN 2} are both satisfied.
\end{example}

\subsection{Natural Transformation}

Let $\mab{a}$ and $\mab{b}$ be two categories then we can view the functors of $\mab{a}$ into $\mab{b}$ as the objects of a category whose morphsisms are given as follows, let $L$ and $M$ be 2 functors of $\mab{a}$ into $\mab{b}$, morphism $H$ of $L$ into $M$ is a rule (called \textbf{natural transformation}) which to each object $X$ of $\mab{a}$ associates a morphism $H_X\in \text{Mor}(L(X),M(X))$ such that for any morphism $f \in \text{Mor}(X,Y)$ the following diagram commutes,

% https://q.uiver.app/#q=WzAsNCxbMCwwLCJMKFgpIl0sWzIsMCwiTShYKSJdLFswLDIsIkwoWSkiXSxbMiwyLCJNKFkpIl0sWzAsMiwiTChmKSIsMl0sWzAsMSwiSF9YIl0sWzEsMywiTShmKSJdLFsyLDMsIkhfWSIsMl1d
\[\begin{tikzcd}
	{L(X)} && {M(X)} \\
	\\
	{L(Y)} && {M(Y)}
	\arrow["{L(f)}"', from=1-1, to=3-1]
	\arrow["{H_X}", from=1-1, to=1-3]
	\arrow["{M(f)}", from=1-3, to=3-3]
	\arrow["{H_Y}"', from=3-1, to=3-3]
\end{tikzcd}\]

And with category of Functors we can talk about isomorphism of functors, a functor is said to be \textbf{representable} if it is isomorphic to a representation functor.

\end{document}

